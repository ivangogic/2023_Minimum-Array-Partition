% Inbuilt themes in beamer
\documentclass{beamer}

% Theme choice:
\usetheme{CambridgeUS}

% Title page details: 
\title{Minimum Array Partition Optimization} 
\author{Daniil Grbić, Ivan Gogić}
\date{\today}

\usepackage[utf8]{inputenc}
\usepackage[T1]{fontenc}

\setlength{\parskip}{12pt}

\begin{document}

\begin{frame}
    \titlepage 
\end{frame}

\section{Opis problema}
\begin{frame}{Opis problema}
    Data je matrica dimenzija $n \times n$ nenegativnih celih brojeva, i broj $p \in \mathbb{N}$.

    Potrebno je izabrati $p-1$ vertikalnih i $p-1$ horizontalnih pregrada koje particionišu matricu na $p^2$ blokova tako minimizujemo sledeći izraz:

    $$
    \max_{\substack{1 \leq i \leq p \\ 1 \leq j \leq p}} \enspace
    \sum_{\substack{v_{i-1} \leq x \leq v_i \\ h_{j-1} \leq y \leq h_j}}
    A[x, y]
    $$

    Odnosno treba da minimuzjemo najveću među sumama elemenata tih blokova.

\end{frame}

\section{Naše metode}
\subsection{Brute Force}
\begin{frame}{Brute Force (BF)}

    \setlength{\parskip}{18pt}

    Algoritam grube sile nam je potreban kako bi utvrdili ispravnost optimizacionih metoda na malim primerima.

    Gledamo sve moguće particije (ima ih $\binom{n}{p}^2$), a sume blokova efikasno računamo koristeći matricu parcijalnih suma (dinamičko programiranje).

    Ubrzanje 3 do 5 puta sečenjem loših rešenja posmatrajući sume proverenih blokova po Dirihleovom principu.

\end{frame}

\subsection{Genetski algoritam}
\begin{frame}[fragile]{Genetski Algoritam}

    \setlength{\parskip}{18pt}

    Genetski algoritam obećava dobre rezultate, ako možemo da smislimo kako da vršimo ukrštanje i mutaciju.

    Van tih funkcija koristimo "školski" genetski algoritam.

    Mutacija je jednostavna: za svaku pregradu u rešenju postoji verovatnoća da se u toku generacije pomeri levo ili desno ukoliko je to mesto prazno. 

    
\end{frame}

\begin{frame}[fragile]{Genetski Algoritam - Ukrstanje}

    Koristimo aritmetički pristup:

    \begin{verbatim}
        roditelj A : 1 3 9     roditelj A : 1 2 3
        roditelj B : 2 5 7     roditelj B : 2 3 4
            dete 1 : 1 4 8         dete 1 : 1 2 3
            dete 2 : 2 4 8         dete 2 : 2 2 4
                                      ...   ...
    \end{verbatim}

    Nastaje problem kada se pregrade preklapaju - to rešavamo ignorisanjem. Rešenja kod kojih se pregrade preklapaju nisu optimalna, i nikad neće na kraju biti izabrana, ali mogu poslužiti kao dobar međukorak.

\end{frame}

\subsection{Simulirano kaljenje i VNS}
\begin{frame}[fragile]{Simulirano kaljenje i VNS}

    \setlength{\parskip}{18pt}

    Klasični algoritmi kaljenja i VNS.

    Kod algoritma kaljenja, male promene postižemo slučajnim pomeranjem pregrada, čime i istražujemo prostor rešenja.

    Kod VNS, promene postižemo slučajnim pomeranjem više pregrada odjednom.

\end{frame}

\section{Rezultati}
\subsection{Generisanje Testova}
\begin{frame}[fragile]{Rezultati - Generisanje Testova}

    \setlength{\parskip}{2pt}

    Testove generišemo po različitim pravilima, naime:
    \begin{description}
        \addtolength{\itemindent}{0.80cm}
        \itemsep0em 
        \item[Random] popunjava matricu brojevima iz $(1, n/2)$
        \item[Linear] gde je $M_{i,j} = i + j + 1$
        \item[Squared] gde je $M_{i,j} = i^2 + j^2 + 1$
    \end{description}

    Za svaki tip testa generišemo po 4 konkretna test primera, sledećih dimenzija:

    \begin{description}
        \addtolength{\itemindent}{0.80cm}
        \itemsep0em 
        \item[Tiny] $n=50$, $p=3$
        \item[Small] $n=80$, $p=6$
        \item[Medium] $n=100$, $p=4$
        \item[Large] $n=500$, $p=10$
    \end{description}

    Svaki test primer propuštamo kroz svaki algoritam po 5 puta, i beležimo najbolje, najgore, i srednje rezultate i vremena izvršavanja.

\end{frame}

\subsection{Random Testovi}
\begin{frame}[fragile]{Rezultati - Random testovi}

    \fontsize{8.2pt}{1pt}

    \begin{verbatim}
    Testcase: tests/random_n100_p4.in
      GENETIC ALGORITHM (GA) [pop_size=100, num_iters=600, elitism_size=20]
        Results  : BEST=17061      WORST=17061      AVG=17061.00  
        Time (s) : BEST=5.06       WORST=5.22       AVG=5.11      
      SIMULATED ANNEALING (SA)
        Results  : BEST=17061      WORST=17061      AVG=17061.00  
        Time (s) : BEST=0.68       WORST=0.68       AVG=0.68      
      VARIABLE NEIGHBOURHOOD SEARCH (VNS) [num_iters=10000]
        Results  : BEST=17061      WORST=17061      AVG=17061.00  
        Time (s) : BEST=4.75       WORST=4.85       AVG=4.80      

    Testcase: tests/random_n500_p10.in
      GENETIC ALGORITHM (GA) [pop_size=100, num_iters=600, elitism_size=20]
        Results  : BEST=326181     WORST=338514     AVG=330243.80 
        Time (s) : BEST=24.44      WORST=25.95      AVG=24.95     
      SIMULATED ANNEALING (SA)
        Results  : BEST=360022     WORST=552450     AVG=424930.40 
        Time (s) : BEST=9.21       WORST=10.39      AVG=9.89      
      VARIABLE NEIGHBOURHOOD SEARCH (VNS) [num_iters=10000]
        Results  : BEST=323192     WORST=417608     AVG=362644.00 
        Time (s) : BEST=24.83      WORST=28.67      AVG=26.29   
    \end{verbatim}

\end{frame}

\subsection{Linear Testovi}
\begin{frame}[fragile]{Rezultati - Linear testovi}

    \fontsize{8.2pt}{1pt}

    \begin{verbatim}
    Testcase: tests/linear_n100_p4.in
      GENETIC ALGORITHM (GA) [pop_size=100, num_iters=600, elitism_size=20]
        Results  : BEST=67760      WORST=68355      AVG=67879.00  
        Time (s) : BEST=5.16       WORST=5.34       AVG=5.23      
      SIMULATED ANNEALING (SA)
        Results  : BEST=67760      WORST=68355      AVG=67879.00  
        Time (s) : BEST=0.71       WORST=0.71       AVG=0.71      
      VARIABLE NEIGHBOURHOOD SEARCH (VNS) [num_iters=10000]
        Results  : BEST=67760      WORST=72500      AVG=69065.00  
        Time (s) : BEST=4.03       WORST=4.87       AVG=4.38      

    Testcase: tests/linear_n500_p10.in
      GENETIC ALGORITHM (GA) [pop_size=100, num_iters=600, elitism_size=20]
        Results  : BEST=1371552    WORST=1621573    AVG=1454183.80
        Time (s) : BEST=23.79      WORST=26.55      AVG=24.81     
      SIMULATED ANNEALING (SA)
        Results  : BEST=1383025    WORST=2073344    AVG=1565182.40
        Time (s) : BEST=9.15       WORST=11.14      AVG=9.76      
      VARIABLE NEIGHBOURHOOD SEARCH (VNS) [num_iters=10000]
        Results  : BEST=1378620    WORST=2168270    AVG=1566251.40
        Time (s) : BEST=21.42      WORST=29.76      AVG=25.80   
    \end{verbatim}

\end{frame}

\subsection{Squared Testovi}
\begin{frame}[fragile]{Rezultati - Squared testovi}

    \fontsize{8.2pt}{1pt}

    \begin{verbatim}
    Testcase: tests/squared_n100_p4.in
      GENETIC ALGORITHM (GA) [pop_size=100, num_iters=600, elitism_size=20]
        Results  : BEST=4808345    WORST=4808345    AVG=4808345.00
        Time (s) : BEST=5.10       WORST=5.16       AVG=5.14      
      SIMULATED ANNEALING (SA)
        Results  : BEST=4807152    WORST=4807152    AVG=4807152.00
        Time (s) : BEST=0.65       WORST=0.70       AVG=0.67      
      VARIABLE NEIGHBOURHOOD SEARCH (VNS) [num_iters=10000]
        Results  : BEST=4808345    WORST=5033479    AVG=4853371.80
        Time (s) : BEST=4.16       WORST=4.76       AVG=4.61      

    Testcase: tests/squared_n500_p10.in
      GENETIC ALGORITHM (GA) [pop_size=100, num_iters=600, elitism_size=20]
        Results  : BEST=522518392  WORST=619537980  AVG=571390189.00
        Time (s) : BEST=23.39      WORST=26.13      AVG=25.28     
      SIMULATED ANNEALING (SA)
        Results  : BEST=533205270  WORST=1038957839 AVG=680165623.40
        Time (s) : BEST=9.39       WORST=11.87      AVG=10.50     
      VARIABLE NEIGHBOURHOOD SEARCH (VNS) [num_iters=10000]
        Results  : BEST=551863221  WORST=851774417  AVG=660668487.00
        Time (s) : BEST=19.00      WORST=25.25      AVG=22.11 
    \end{verbatim}

\end{frame}

\subsection{Zakljucak}
\begin{frame}[fragile]{Zakljucak}

    \fontsize{8.2pt}{1pt}

    Pokazali smo da je moguće uspešno koristiti optimizacione algoritme za rešavanje Minimum Array Partition problema.
    
    Genetiski algoritam daje najkonzistentnije rezultate, VNS je tu odmah iza njega.

    Simulirano kaljenje je dobra alternativa kada možemo da prihvatimo neoptimalno rešenje, a sprečava nas vreme.

\end{frame}


\end{document}